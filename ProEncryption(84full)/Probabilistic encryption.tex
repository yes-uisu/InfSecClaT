\documentclass[]{article}
\usepackage[UTF8]{ctex}
\usepackage{graphicx}
\usepackage{amsmath}
\newtheorem{definition}{Definition}
\newtheorem{theorem}{Theorem}

%opening
\title{概率加密\footnote{原文:Goldwasser S ,  Micali S . Probabilistic Encryption[J]. Journal of Computer and System Sciences, 1984, 28(2):270-299.}}
\author{作者:S Goldwasser,S Micali\\
\small{译者:李晓峰,北京联合大学智慧城市学院\footnote{译者email:cy\_lxf@163.com,译文来自于译者发起的“信息安全经典翻译”开源项目https://gitee.com/uisu/InfSecClaT}}
}

\begin{document}

\maketitle

\begin{abstract}
介绍了一种新的数据加密概率模型。对于该模型,在适当的复杂度假设下,证明了对于具有多项式有界计算资源的对手来说,从密码文本中提取任何关于明文的信息平均来说是困难的。该证明适用于具有任何概率分布的任何消息空间。给出了该模型的第一个实现。在二次剩余模复合数因式分解是困难问题的假设下,证明了该实现的安全性。
\end{abstract}

\section{引言}


\subsection{Deterministic Encryption: The Trapdoor Function Model}

\subsection{Basic Objections to the Trapdoor Function Model}


\subsection{Probabilistic Encryption: The New Model}



\subsection{Concrete Implementation of the New Model}



\subsection{Related Work}


\section{SURVEY OF PUBLIC KEY CRYPTOSYSTEMS BASED ON TRAPDOOR FUNCTIONS}


\subsection{What Is a Public Key Cryptosystem?}




\subsection{The RSA Scheme and the Rabin Scheme}



\subsection{Objections to Cryptosystems Based on Trapdoor Functions}



\section{UNAPPROXIMABLE TRAPDOOR PREDICATES}

\subsection{Quadratic Residuosity as a UTP}

\section{PUBLIC KEY CRYPTOSYSTEMS AND PROBABILISTIC	PUBLIC KEY CRYPTOSYSTEMS}

\section*{致谢}

\section*{参考文献}

\end{document}
