\documentclass[]{article}
\usepackage[UTF8]{ctex}
\usepackage{amsmath}

\usepackage{amsmath}
\usepackage{graphicx}
\newtheorem{theorem}{Theorem}
\renewcommand{\thesection}{\Roman{section}}

%opening
\title{***}
\author{ABC, \\
{\small  翻译:李晓峰(cy\_lxf@163.com)}\footnote{译者目前为北京联合大学智慧城市学院信息安全老师。}\footnote{译文来自于经典文献翻译项目https://gitee.com/uisu/InfSecClaT,欢迎大家加入经典翻译项目,为更多的人能够获取这些经典文献所传递信息做一点贡献。}\\
{\small  V1.0}
}

\usepackage{hyperref} %生产书签

\begin{document}
	
	\maketitle
	
	\begin{abstract}
		本文讨论了密码学现代发展的两个方面。远程处理的广泛应用促使开发新型密码系统,这些系统最大限度地减少了对安全密钥分发通道的需求,提供了等同于书面签名的功能。本文提出了解决这些当前未解决问题的方法。同时,本文也讨论了通信和计算理论开始提供解决长期存在的密码学问题的工具。
	\end{abstract}

	\section{引言}
	
	
	
	
	
	\section*{参考文献}
    1. R. Merkle, “Secure communication over an insecure channel,”submitted to Communications of the ACM.\par
	2. D. Kahn, The Codebreakers, The Story of Secret Writing. New
	York: Macmillan, 1967.\par
	3. C. E. Shannon, “Communication theory of secrecy systems,” Bell
	Syst. Tech. J., vol. 28, pp. 656-715, Oct. 1949.\par
	4. M. E. Hellman, “An extension of the Shannon theory approach to
	cryptography,” submitted to IEEE Trans. Inform. Theory, Sept.
	1975.\par
	5. W. Diffie and M. E. Hellman, “Multiuser cryptographic techniques,”
	presented at National Computer Conference, New York, June 7-10,
	1976.\par
	6. D. Knuth, The Art of Computer Programming, Vol. 2, Semi-
	Numerical Algorithms. Reading, MA.: Addison-Wesley, 1969.\par
	7. --, The Art of Computer Programming, Vol. 3, Sorting and
	Searching. Reading, MA.: Addison-Wesley, 1973.\par
	8. S. Pohlig and M. E. Hellman, “An improved algorithm for com-
	puting algorithms in GF(p) and its cryptographic significance,”
	submitted to IEEE Trans. Inform. Theorv.\par
	9. M. V. Wilkes, Time-Sharing Computer Systems. New York: El-
	sevier, 1972.\par
	l0. A. Evans, Jr., W. Kantrowitz, and E. Weiss, “A user authentication
	system not requiring secrecy in the computer,” Communications
	of the ACM, vol. 17, pp. 437-442, Aug. 1974.\par
	11.G. B. Purdy, “A high security log-in procedure,” Communications
	of the ACM, vol. 17, pp. 442-445, Aug. 1974.\par
	12.W. Diffie and M. E. Hellman, “Cryptanalysis of the NBS data en-
	cryption standard” submitted to Computer, May 1976.\par
	13.A. V. Aho, J. E. Hopcroft, and J. D. Ullman, The Design and
	Analysis of Computer Algorithms. Reading, MA.: Addison-
	Wesley, 1974.\par
	14.R. M, Karp, “Reducibility among combinatorial problems,” in
	Complexity of Computer Computations. R. E. Miller and J. W.
	Thatcher, Eds. New York: Plenum, 1972, pp. 855104.\par
\end{document}

