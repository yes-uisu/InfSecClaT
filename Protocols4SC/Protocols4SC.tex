\documentclass[]{article}
\usepackage[UTF8]{ctex}
\usepackage{amsmath}


\newtheorem{theorem}{Theorem}

%opening
\title{安全计算协议}
\author{姚期智\\
	加利福尼亚大学伯克利分校\\
{\small  翻译:李晓峰(cy\_lxf@163.com)}\\
{\small  译文来自于经典文献翻译项目https://gitee.com/uisu/InfSecClaT}\\
{\small 译者单位:北京联合大学智慧城市学院}
}

\begin{document}
	
	\maketitle
	
	\begin{abstract}
		此文是对姚期智老师Protocols for secure computations文章的翻译。
	\end{abstract}
	
	\section{引言}
	
	\section{安全计算的统一视图}
	
	\section{确定性计算}
	\subsection{百万富翁问题的解决方案}
	在这个摘要中,我们将详细描述我们所拥有的三种解决方案中的一种。\par
	
	为了明确起见,设Alice有$i$百万,Bob有$j$百万,并且$1\leq i,j \leq 10$,我们需要一个协议来判断“$i$是否小于$j$”,并且最后我们只能得到这个信息(也就是说没有获得多余的有关$i$,$j$的信息),设M是所有N比特非负整数集合,$Q_N$是所有M到M的1-1满射函数(onto function)集合,$E_a$是Alice的公钥,从$Q_N$中随机选取的。\footnote{译者注:$E_a$逆函数$D_a$只有Alice知道,对于Bob同样有$E_b$和$D_b$。}\par
	
	协议处理过程如下:\par
	
	\begin{enumerate}
		\item Bob取一个N比特随机数$x$,计算$k=E_a(x)$。
		\item Bob把$k-j$发给A.\footnote{译者注:原文中是$k-j+1$}。
		\item Alice计算$y_u=D_a(k-j+u),u=1,2,\ldots,10$。\footnote{译者注:这里计算的是$y_1,\ldots,y_j,\ldots,y_10$,特别注意$y_j=D_a(k-j+j)=D_a(k)=x$。而此时Alice并不知道这里就是Bob的值。}
		\item Alice随机选一个N/2比特随机素数$p$,计算$z_u=y_u \pmod{p},u=1,2,\ldots,10$,如果所有$z_u$在模$p$下至少相差2,则停止,否则产生新的$p$,直至条件成立,也就是说直至$|z_u - z_v| \geq 2,u,v\in\{1,2,\ldots10\},u\neq v$。\footnote{译者注:此处要求相差为2,是因为在下面的步骤中通过加一改变了原数值,而这种改变应该是能够与原值进行区分的。}
		\item Alice把$p$和10个数:$z_1,z_2,\ldots,z_{i}$和$z_{i+1}+1,z_{i+2}+1,\ldots,z_{10}+1$发送给Bob。以上数都是在模$p$的运算。
		\item Bob取Alice发来的第$j$个数$w$,如$w=x \pmod{p}$,那么$i\geq j$,否则$i < j$。\footnote{译者注:如果$j$处的数没有变化,则说明$j$是在$i$前面或者就是$i$,没有+1。}
		\item Bob告诉Alice比较结果。
	\end{enumerate}
	
	该协议显然能够使Alice和Bob正确地决定谁是更富有的人。为了证明该协议符合他们无法获取对方财富的任何更多信息的要求,我们将在第3.2节中给出一个精确的模型。在这里,我们将非正式地论证为什么该要求能够得到满足。\par
	
	首先,Alice对Bob的财富j一无所知,除了Bob告诉她的最终结果所隐含的j约束,因为来自Bob的唯一其他信息是Bob知道k−j + 1到k−j + 10之间某个s的$D_a(s)$的值。由于函数$E_a$是随机的所有10种可能性都是等概率的。\par
	
	Bob知道什么?他知道$y_j$(也就是x)因此也知道$z_j$。然而,他没有关于其他$z_u$值的信息,并且通过查看Alice发送给他的数字,他无法判断它们是$z_u$还是$z_u + 1$。\par
	
	这还没有结束争论,因为Alice或Bob可能会试图通过更多的计算来计算对方的价值。例如,Bob可能会尝试随机选择一个数字t并检查$E_a(t)=k−j + 9$是否成立;如果他成功了,他就知道$y_9$的值是t,并且知道$z_9$的值,这样他就可以知道是否$i\geq 9$。如果$i\geq j$是前一个结论的结果,那么这将是Bob不应该发现的额外信息。因此,我们还必须在正式定义中包括,参与者不仅没有通过协议指定的交换获得信息,而且他们也不能在合理的时间内执行计算以获得该信息。在3.2节给出的正式定义中,我们将对此进行精确的定义。\par
	
	人们可能已经注意到,在这个过程中,某些方面可能会通过偏离商定的协议而作弊。例如,Bob可能在最后一步对爱丽丝撒谎,告诉爱丽丝错误的结论。是否有一种设计协议的方法,使得成功作弊的机会变得非常小,而不暴露i和j的值?我们将在3.3节展示这是可能的。(请注意,这是一个比Shamir等人[5]在心理扑克协议中使用的可验证性要求更强的要求。)\par
	
	针对百万富翁的问题,我们有另外两种基于不同原则的解决方案。第一个方案假设Alice和Bob各自拥有一个私有的单向函数,其中这些函数满足交换性,即$E_a E_b(x)=E_bE_a(x)$。另一个方案利用Goldwasser和Micali[2]发明的概率加密方法。
	
	\subsection{一般性问题的模型}
	Alice有个秘密数i,Bob有个秘密数j,假设Alice有一个公共单向函数$E_a$,其逆函数是$D_a$,逆函数只有Alice知道,对于Bob同样有函数$E_b,D_b$,假设$E_a,E_b$相互独立并且是从$Q_N$中随机选取,$Q_N$是N比特整数的1-1满射函数集合,下面我们精确地描述Alice和Bob如何通过协议$\Lambda$计算$f(i,j)$。\par
	Alice和Bob交替给对方发送字符串。\par
	Bob每次发送完成,Alice检查她所拥有的信息:\\
	1、字符串序列$\alpha_1,\alpha_2,\ldots,\alpha_t$\\
	2、这些字符串之间的关系,比如$E_b(\alpha_3)=\alpha_9$,$\alpha_8$有奇数个1.\\
	3、根据Alice和Bob至此已经传输过的比特,协议说明Alice如何计算隐私字符串$\alpha_{t+1},\alpha_{t+2},\ldots,\alpha_s$,此处每一个新的字符串$\alpha_u,u\in \{t+1,\ldots,s\}$都是以前字符串的函数,或者说新字符串都是这样的形式$E_a(y),E_b(y)$或$D_a(y)$,此处y是Alice已经获得的字符串。A随机选择使用哪个函数,例如,Alice投币决定使用$E(4)$或者计算$\alpha_2+3\alpha_8$。\\
	4、Alice计算完后,她将发一个字符串给Bob,选择发送哪个字符串也是随机的。\par
	Bob收到字符串后,他也按Alice的方法计算一些字符串,并且根据协议发送一个字符串。\par
	
	Alice和Bob达成一致,当收到一个特殊的字符时,协议执行结束,这时,协议有一条指令,就是每个参与者都秘密计算函数f的值,最后,在协议中,我们要求Bob和Alice计算E和D的数量受$O(N^k)$的限制,此处k是一个事先选择好的整数。\par
	
	\textbf{隐私限制(Privacy Constraint)}\par
	
	设$\epsilon,\delta >0$, $f(i,j)$函数值为0或1,假定初始时所有(i,j)取值可能性都是一样的,并且假定Bob和Alice根据协议忠实第计算,最后Alice原则上可以根据她计算的函数值v和她拥有的字符串,计算j值的概率分布$p_i(j)$.一个协议如果满足以下条件,我们就说此协议满足$(\epsilon,\delta)$隐私限制:\\
	1.$p_i(j)=\frac{1}{\|G_i\|} (1+O(\epsilon)),j\in G_i$,此处$G_i$是使$f(i,j)=v$等式成立的所有j组成的集合,如果$j\notin G_i$,则$p_i(j)=0$.\\
	2.如果Alice之后尝试执行更多计算计算E和D,但计算的次数不超过$ O(N^k)$ 次,那么她会以至少 $1 − \delta$ 的概率仍然得到 j 上的上述概率分布。\\
	3.对于Bob也有以上同样要求.
	\par
	
	\begin{theorem}
		对于任何$\epsilon,\delta >0$和任何函数f,存在一个用于计算f的协议满足$(\epsilon,\delta)$隐私限制。
	\end{theorem}

	\subsection{增加的需求}
	\textbf{复杂性(complexity)}\par
	文章中给出的百万富翁算法并不实用,因为决定i,j范围的n如果很大,那么传输的比特也会很多,因为传输的比特数与n是一个正比关系,那么一个有意思的问题就出现了:\\
	对于满足$(\epsilon,\delta)$隐私限制的用于计算f的任一协议来说,所学传输的最小比特数是多少?
	\par
	可以想象,在没有因私限制时,有一些函数很容易计算,但是当有额外的隐私限制时,就变得很不容易。幸运的是,
	我们可以证明事实并非如此。假设$\Lambda$是一个协议,当使用此协议时,Alice和Bob之间传输的最大比特数记为$T(\Lambda)$.\par
	
	\begin{theorem}
		设$1>\epsilon,\delta >0$,f(i,j)是一个0-1函数,如果f可以被一个规模为C(f)的布尔电路计算,那么这里就有一个计算f的协议$\Lambda$满足$(\epsilon,\delta)$隐私限制,并且$T(\Lambda)=O(C(f)\log \frac{1}{\epsilon \delta})$.
	\end{theorem}

	事实上,如果f可以被一个图灵机在时间S内计算,那么这个协议可以被实现,以至于Alice和Bob都有图灵机算法来执行这个协议在$O(S\log(\frac{1}{\epsilon \delta}))$.
	\par
	
	\textbf{相互怀疑的参与者(Mutually-Suspecting Participants)}\par
	
	\subsection{应用}
	
	\section{概率计算}
	
	\section{m方情况的一般化描述}
	
	\section{什么不能做}
	
	
\end{document}

