\documentclass[]{article}
\usepackage[UTF8]{ctex}
\usepackage{amsmath}

\usepackage{amsmath}
\usepackage{graphicx}
\newtheorem{theorem}{Theorem}
\renewcommand{\thesection}{\Roman{section}}

%opening
\title{转变中的密码学\\
Cryptology in Transition}
\author{ABRAHAM LEMPEL, \\
\small{Department of Electrwal Engmeertng, Technton--Israel Instttute of Technology, Hatfa, Israel}
{\small  翻译:李晓峰(cy\_lxf@163.com)}\footnote{译者目前为北京联合大学智慧城市学院信息安全老师。}\footnote{译文来自于经典文献翻译项目https://gitee.com/uisu/InfSecClaT,欢迎大家加入经典翻译项目,为更多的人能够获取这些经典文献所传递信息做一点贡献。}\\
{\small  V1.0}
}

\usepackage{hyperref} %生产书签

\begin{document}
	
	\maketitle
	
	\begin{abstract}
		本综述的重点是最近提出的实现方案,以及与密码复杂性
		有关的公钥密码系统问题的新概念。此外,还简要概述了经典密码学,当今密码学的基本原则,以及现在官方
		的DES加密标准。
	\end{abstract}

	\section*{前言}
	这项工作旨在向读者传达公钥密码学目前
	所处的过渡阶段和不断发展的技术状态。
	这种过渡是多方面的:在相关范围内,它
	已经从处理军事和外交通信的政府垄断演
	变为一般商业的主要关注,特别是银行业,
	以及最近的广大公众。它的技术已经从纸
	笔和各种机械设备扩展到大型、高速的电
	子计算机。它的安全重点也从统计的不确
	定性转变为计算的复杂性。最后,但并非
	最不重要的是,在概念上它已经从传统的
	pri- vat -key 方案发展到公钥密码系统——
	提供即时隐私和双向认证的术语。
	
	\section*{引言}
	
	\section{古典密码}
	
	\subsection{凯撒密码}
	
	\subsection{简单替换}
	
	\subsection{多字母密码}
	
	\subsection{换位密码}
	
	\subsection{乘积密码}
	
	\section{现代密码的基本原理}
	
	\subsection{流密码}
	
	\subsection{分组密码}
	
	\subsection{DES}
	
	\section{公钥密码}
	
	\subsection{Rivest-Shamir-Adleman(RSA)方案}
	
	\subsection{Merkel-Hellman(MH)方案}
	
	\subsection{McEliece方案}
	
	\subsection{Graham-Shamir(GS)方案}
	
	\subsection{纯签名方案}
	
	\section{错综复杂的密码复杂度}
	
	\subsection*{例子:一个容易破解的NP-完全的密码}
	这个例子是由作者 Shimon Even 和 Yacov
	Yacobi 共同推导出来的。它演示了一个密
	码,即使在选择的明文攻击下,破解其密
	钥的问题也是NP-完全的。然而,给定足
	够的已知明文,以接近1的概率,可以将破解
	密钥规约为在 n 个未知数中求解 n 个
	独立线性方程的简单问题,其中 n 是密钥
	位的数量。\par
	
	
	该方案是一种传统的私钥分组密码,其
	总体结构如图 1 所示。密钥长度为 n 位 ,
	$K = (x_l x_2 \ldots  x_n)$ , 消息是长度为 m 的二进
	制块,其中 $m = [log_2(1 + \sum_{j=1}^{n} a_j)]$, $A =
	(a_l a_2 \ldots a_n) $是一个任意的具有正整数分量
	$a_j$的背包,假设密码分析者知道。要获
	得消息 M 的密文 C,请按照以下步骤
	进行:\par
	
	\begin{itemize}
		\item 在本地生成一个特别随机(ad hoc random)二进制向量R;
		\item 计算$s=A(K\oplus R)^T$;
		\item 集合$C=(M\oplus S,R)$,此处$S=binary(s)$
	\end{itemize}
	
	注意密文长度是$m+n$,$M\oplus S$的$m$位后面跟着$R$的$n$位,并且对于每个
	$m$ 生成一个新的 $R$。解密也很简单:因为合法
	的接收者知道 $K$,他可以把 $K$ 加到接收到的
	$R$中,计算出 $s$,从中他可以得到$S$,从而得
	到 $M$。\par
	
	从密码分析者的角度来看,最糟糕的情
	况是,对于他所有已知的明文,特设向量(ad hoc vector)
	R 保持不变。然后,在已知和选择明文攻
	击下,破解密钥需要解决方程 $s=A(K\oplus R)^T$,给定 s, A 和 R 求 K。当然,这相
	当于解决NP-完全的背包问题。\par
	
	一个可能性更大的事件是,给定足够多
	的已知明文,密码分析师将有 n 对 $(M_i,C_i)$ ,
	其中 $C_i= (M_i\oplus S_i, R_i)$ , $i = 1,2 \ldots n$,
	使得 n 个向量 $U_i = 1^n- 2R_i$在实数上是线性
	无关的($1^n$ 是 n 个 l的向量)。对于每一个 $i= 1,2,\ldots ,M$,我们有
	\[K\oplus R_i=K+R_i-2(K * R_i)\]
	其中*表示分量乘法(componentwise multiplication)。因此:
	\[K\oplus R_i=R_i+K * U_i\]
	并且
	\begin{align*}
		s_i & =A(K\oplus R_i)^T\\
	        & =A(R_i+K*U_i)^T\\
	        & =AR_i^T+A(K*U_i)^T\\
	        & =AR_i^T+(U_i*A)K^T
	\end{align*}
	
	
	令$t_i=s_i-AR_i^T$,将 n 个方程写成矩阵形式,我们得到
	
	\begin{equation*}
		\begin{bmatrix}
		t_1 \\
		t_2 \\
		\dots \\
		t_n
		\end{bmatrix}=
		\begin{bmatrix}
		U_1 \\
		U_2 \\
		\dots \\
		U_n
		\end{bmatrix}
		\begin{bmatrix}
		a_1 & 0 & \dots &0 \\
		0 & a_2 & \dots &0 \\
		\dots & \dots & \dots &\dots \\
		0 & 0 & \dots &a_n \\
		\end{bmatrix}
		\begin{bmatrix}
		x_1 \\
		x_2 \\
		\dots \\
		x_n
		\end{bmatrix}
	\end{equation*}
	
	由于 $U_i$是独立的,并且对于所有 $j$ 都有
	$a_j> 0$,可以很容易地解出密钥分量$x_i$。
	可以证明(见 LEMP78,附录 1)$N\geq n$ 对
	$(M, C)$产生 $n$ 个线性无关的 $U$, 对于 $N = n$的概率下
	界约为$1/3$;当 $N = n + 1$ 时,下
	界加倍,随着 $N - n$ 的增加,下界迅速趋
	近于 1。\par
	
	
	这个例子并不是要以任何方式减损作为加密方案基础的难题的
	潜在有用性。我们的目的只是提醒读者,
NP-完备的复杂性度量,以及在 NPC类
	中困难问题的公认难度并未解决,就加密
	复杂性而言可能超出了上面复杂度的上下文。通过将
	Merkle-Hellman 方案与我们的示例进行比
	较,进一步强调了所涉及的复杂性。这两
	种方案都是基于背包问题,虽然还不清楚
	破解MH方案是否是NPC,但我们的例子
	中的方案肯定是这样的。然而,破解后者
	非常简单,而破解 MH 方案的可行方法尚
	未找到。\par
	
	还应该指出,示例方案的大部分(如果不
	是全部)弱点是由于 $C$ 对 $M$ 的线性依赖。
	长期以来,密码中的线性一直被认为是密
	码学家的诅咒和密码分析者的福音。我们在第 3 节的
	末尾讨论了这个问题,就是具有某种线性的方案中的隐私和身份
	验证之间的权衡时。\par
	
	很容易修改我们示例的方案,使其密码
	分析像任何已知的一样困难。例如,可以
	用 $M^s \pmod{pq}$ 代替 $M \oplus S$ ,就像在 RSA 方
	案中一样,或者用 DES(M, S)代替$M \oplus S$,
	也就是说,通过将 DES 方案应用于 M, 而 S
	作为 DES 密钥。我们不提这一点,
	是为了提出另一个方案,其难解性是一个
	猜想问题。如前所述,我们认为需要的是
	努力建立一些在密码复杂性背景下真正有
	效的标准,然后才尝试发明符合这些标准
	的方案。正如 RABI77 中所指出的那样,
	我们离实现这一目标还很遥远。
	
	\section*{致谢}
	作者很高兴地感谢我在准备这项调查时得到的帮助。
	特别感谢Martin Cohn在内容和风格上
	提 供 的 宝 贵 帮 助 , 感 谢 Martin 	Hellraan 帮 助 修 改Sperry的原 始报 告, 并感 谢
	Ron Graham 和Adl Shamir 允许描述他们的未发表的方案。
	作者也感谢与Len Adleman,Shimon Even,Leland Gardner,Martin Hellman, Ralph Merkle, Nick Plppenger, Michael Rabin,
	Ron Rivest, Adi Shamir, Ned Sloane, Shmuel Wmograd, and Yacob Yacobi 的有益讨论
	
	\section*{参考文献}
    \par
\end{document}

