\documentclass[]{article}
\usepackage[UTF8]{ctex}
\usepackage{amsmath}

\usepackage{amsmath}
\usepackage{graphicx}
\newtheorem{theorem}{Theorem}
\renewcommand{\thesection}{\Roman{section}}

%opening
\title{如何产生和交换秘密}
\author{姚期智, 普林斯顿大学计算机科学系\\
{\small  翻译:李晓峰(cy\_lxf@163.com)}\footnote{译文来自于经典文献翻译项目https://gitee.com/uisu/InfSecClaT,欢迎大家加入经典翻译项目,为更多的人能够获取这些经典文献所传递信息做一点贡献。}\\
{\small  V1.0}
}

\usepackage{hyperref} %生产书签

\begin{document}
	
	\maketitle
	
	\begin{abstract}
		本文介绍了一种控制密码协议设计中知识传递过程的新工具。它被应用于解决一类一般的问题,其中包括文献中大多数的两方密码问题。具体来说,我们展示了双方A和B如何交互地生成一个随机整数$N = pq$,使得其秘密,即质因数$(p,q)$,对任何一方都是单独隐藏的,但如果需要,可以共同恢复。利用该模型,可以给出一个协议,使具有私有值i和j的双方在具有最小知识转移和强公平性( strong fairness property)的前提下,计算任意多项式可计算函数$f(i,j)$和$g(i,j)$。作为一个特例,A和B可以交换一对秘密$S_A$, $S_B$,例如分解一个整数和图中的一个哈密顿回路,使得当且仅当$S_B$能被A计算时,$S_A$也能被B计算。所有这些结果都是在假定大数分解问题是计算困难的情况下证明的。
	\end{abstract}

	\section{引言}
	
	
	\section{术语}
	
	\section{生成一个秘密}
	
	\section{交换一个秘密}
	
	\section{通用计算}
	
	
	\section{正确性}
	
	\section*{参考文献}
    [ ACGS ] W . Alexi , B . Chor , O . Goldreich ,    and C . P . Schnorr ,    ” RSA / Rabin bits are l / 2 - f l / 2 poly ( log n ) secure , ”    Proceedings of 25 th Annual IEEE Symposium on Foundations of   Computer Science , 1984 , 449 - 457 .\par
    [ Bl ] M . Blum , " Coin flipping by phone , ”  COMPCON ( 1982 ) ,133 - 137 .\par
    [ B 2 ] M . Blum , " How to exchange ( secret ) keys , ”  ACM Transactions on Computer Systems 1 ( 1983 ) , 175 - 193 .\par
    [ BS ] M . Blum and S . Goldwasser , ” An efficient probabilistic PKCS as secure as factoring , ” Proceedings of Crypto 84 ,
    1984\par
    [ C ] R . Cleve , " Limits on the security of coin flips when half of  the processors are faulty , ” Proceedings of 18 th Annual ACM  Symposium on Theory of Computing , 1986 , 364 - 369 .\par
    [ FMRW ] M . Fischer , S . Micali , C . Rackoff , and D . Wittenberg , ” An oblivious transfer protocol , ” 1985 , to appear .\par
    [ GHY ] Z . Galil , S . Haber , and M . Yung , ” A private interactive test  of a Boolean predicate and minimum - knowledge public - key cryptosystems , ” Proceedings of 26 th Annual IEEE Symposium on Foundations of Computer Science , 1985 , 360 - 371 .\par
    [ GM ] S . Goldwasser and S . Micali , " Probabilistic encryption and how to play mental piker keeping secret all partial information , ” Proceedings of 14 th Annual ACM Symposium on Theory of Computing , 1982 , 365 - 377 .\par
    [ GMR ] S . Goldwasser , S . Micali , and C . Rackoff , " The knowledge complexity of interactive proof systems , ” Proceedings of  17 th Annual ACM Symposium on Theory of Computing ,   1985 , 291 - 304 .\par
    [ GMW ] 0 . Goldreich , S . Micali , and A . Wigderson , " Proofs that    yield nothing but their validity and a methodology of cryptographic protocol design , ” Proceedings of 27 th Annual  IEEE Symposium on Foundations of Computer Science ,    1986\par
    [ HS ] J . Hastad and A . Shamir , " The cryptographic security of  truncated linearly related variables , ” Proceedings of 17th  Annual ACM Symposium on Theory of Computing , 1985 ,  356 - 362 .\par
    [ LMR ] M . Luby , S . Micali , and C . Rackoff , " How to simultaneously exchange a secret bit by flipping a symmetrically
    based coin , ” Proceedings of 24 th Annual IEEE Symposium on Foundations of Computer Science , 1985 , 11 - 22 .\par
    [ R ] M . Rabin , " How to exchange secrets , ” 1981 , unpublished  manuscript .\par
    [ SRA ] A . Shamir , R . Rivest , and L . Adleman , " Mental Poker , ”   MIT Technical Report , 1978 \par
    [ T ] T . Tedrick , ” How to exchange half a bit , ” Crypto ’ 88 .\par
    [ VV ] U . Vazirani and V . ‘ Vazirani , " Trapdoor pseudo - random   number generators , with applications to protocol design , ”  Proceedings of   24 th Annual IEEE Symposium on Foundations of Computer Science , 1985 , 23 - 30 .\par
    [ Yl ] A . Yao , " Protocols for secure computations , ” ( extended abstract ) Proceedings of 21 st Annual IEEE Symposium
    Foundations of Computer Science , 1982 .\par
    [ Y 2 ] A . Yao , " Protocols for secure computations , ” in preparation .\par
    [ Y 3 ] A . Yao , " Theory and applications of trapdoor functions , ”Proceedings of 21 st Annual IEEE Symposium on Founda -
    tions of Computer Science , 1982 \par
\end{document}

