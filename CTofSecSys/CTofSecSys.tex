\documentclass[]{article}
\usepackage[UTF8]{ctex}
\usepackage{amsmath}

\newtheorem{theorem}{Theorem}

%opening
\title{安全计算协议}
\author{姚期智\\
	加利福尼亚大学伯克利分校\\
{\tiny  翻译:李晓峰}\\
{\tiny 翻译者单位:北京联合大学智慧城市学院}
}

\begin{document}
	
	\maketitle
	
	\begin{abstract}
		此文是对姚期智老师Protocols for secure computations文章的翻译。
	\end{abstract}
	
	\section{引言}
	
	\section{安全计算的统一视图}
	
	\section{确定性计算}
	\subsection{百万富翁问题的解决方案}
	\subsection{一般性问题的模型}
	Alice有个秘密数i,Bob有个秘密数j,假设Alice有一个公共单向函数$E_a$,其逆函数是$D_a$,逆函数只有Alice知道,对于Bob同样有函数$E_b,D_b$,假设$E_a,E_b$相互独立并且是从$Q_N$中随机选取,$Q_N$是N比特整数的1-1满射函数集合,下面我们精确地描述Alice和Bob如何通过协议$\Lambda$计算$f(i,j)$。\par
	Alice和Bob交替给对方发送字符串。\par
	Bob每次发送完成,Alice检查她所拥有的信息:\\
	1、字符串序列$\alpha_1,\alpha_2,\ldots,\alpha_t$\\
	2、这些字符串之间的关系,比如$E_b(\alpha_3)=\alpha_9$,$\alpha_8$有奇数个1.\\
	3、根据Alice和Bob至此已经传输过的比特,协议说明Alice如何计算隐私字符串$\alpha_{t+1},\alpha_{t+2},\ldots,\alpha_s$,此处每一个新的字符串$\alpha_u,u\in \{t+1,\ldots,s\}$都是以前字符串的函数,或者说新字符串都是这样的形式$E_a(y),E_b(y)$或$D_a(y)$,此处y是Alice已经获得的字符串。A随机选择使用哪个函数,例如,Alice投币决定使用$E(4)$或者计算$\alpha_2+3\alpha_8$。\\
	4、Alice计算完后,她将发一个字符串给Bob,选择发送哪个字符串也是随机的。\par
	Bob收到字符串后,他也按Alice的方法计算一些字符串,并且根据协议发送一个字符串。\par
	
	Alice和Bob达成一致,当收到一个特殊的字符时,协议执行结束,这时,协议有一条指令,就是每个参与者都秘密计算函数f的值,最后,在协议中,我们要求Bob和Alice计算E和D的数量受$O(N^k)$的限制,此处k是一个事先选择好的整数。\par
	
	\textbf{隐私限制(Privacy Constraint)}\par
	
	设$\epsilon,\delta >0$, $f(i,j)$函数值为0或1,假定初始时所有(i,j)取值可能性都是一样的,并且假定Bob和Alice根据协议忠实第计算,最后Alice原则上可以根据她计算的函数值v和她拥有的字符串,计算j值的概率分布$p_i(j)$.一个协议如果满足以下条件,我们就说此协议满足$(\epsilon,\delta)$隐私限制:\\
	1.$p_i(j)=\frac{1}{\|G_i\|} (1+O(\epsilon)),j\in G_i$,此处$G_i$是使$f(i,j)=v$等式成立的所有j组成的集合,如果$j\notin G_i$,则$p_i(j)=0$.\\
	2.如果Alice之后尝试执行更多计算计算E和D,但计算的次数不超过$ O(N^k)$ 次,那么她会以至少 $1 − \delta$ 的概率仍然得到 j 上的上述概率分布。\\
	3.对于Bob也有以上同样要求.
	\par
	
	\begin{theorem}
		对于任何$\epsilon,\delta >0$和任何函数f,存在一个用于计算f的协议满足$(\epsilon,\delta)$隐私限制。
	\end{theorem}

	\subsection{增加的需求}
	\textbf{复杂性(complexity)}\par
	文章中给出的百万富翁算法并不实用,因为决定i,j范围的n如果很大,那么传输的比特也会很多,因为传输的比特数与n是一个正比关系,那么一个有意思的问题就出现了:\\
	对于满足$(\epsilon,\delta)$隐私限制的用于计算f的任一协议来说,所学传输的最小比特数是多少?
	\par
	可以想象,在没有因私限制时,有一些函数很容易计算,但是当有额外的隐私限制时,就变得很不容易。幸运的是,
	我们可以证明事实并非如此。假设$\Lambda$是一个协议,当使用此协议时,Alice和Bob之间传输的最大比特数记为$T(\Lambda)$.\par
	
	\begin{theorem}
		设$1>\epsilon,\delta >0$,f(i,j)是一个0-1函数,如果f可以被一个规模为C(f)的布尔电路计算,那么这里就有一个计算f的协议$\Lambda$满足$(\epsilon,\delta)$隐私限制,并且$T(\Lambda)=O(C(f)\log \frac{1}{\epsilon \delta})$.
	\end{theorem}

	事实上,如果f可以被一个图灵机在时间S内计算,那么这个协议可以被实现,以至于Alice和Bob都有图灵机算法来执行这个协议在$O(S\log(\frac{1}{\epsilon \delta}))$.
	\par
	
	\textbf{相互怀疑的参与者(Mutually-Suspecting Participants)}\par
	
	\subsection{应用}
	
	\section{概率计算}
	
	\section{m方情况的一般化描述}
	
	\section{什么不能做}
	
	
\end{document}

