\documentclass[]{article}
\usepackage[UTF8]{ctex}
\usepackage{amsmath}

\usepackage{amsmath}
\usepackage{graphicx}
\newtheorem{theorem}{Theorem}
\renewcommand{\thesection}{\Roman{section}}

%opening
\title{密码学和复杂性理论\\
cryptology and complexity theories}
\author{G. Ruggiu \\
{\small  翻译:李晓峰(cy\_lxf@163.com)}\footnote{译者目前为北京联合大学智慧城市学院信息安全老师。}\footnote{译文来自于经典文献翻译项目https://gitee.com/uisu/InfSecClaT,欢迎大家加入经典翻译项目,为更多的人能够获取这些经典文献所传递信息做一点贡献。}\\
{\small  V1.0}
}

\usepackage{hyperref} %生产书签

\begin{document}
	
	\maketitle
	
	\begin{abstract}
		复杂性理论最近被用来做为密码机性能评估的基础,与香农模型相比,在随机性概念上有新的亮点,但需要强调的是,统计学的观点仍然更加可靠。
	\end{abstract}


    复杂性理论最近被提出作为评估密码系统性能的基础。我们将在这个简短的调查中介绍用于连接这两个概念的不同方法。
    
    复杂性理论是相当新的理论,其动机是分析算法的效率。它们的主要特点是它们是处理非常通用的算法的非常通用的理论:它们最具体的结果给出了关于算法的渐近行为的一些信息。
    
    密码学的核心问题是对保密系统的安全性进行评估,即系统如何对密码分析免疫。当这种密码分析成为可能时,这种评估必须衡量破解方案所需的时间和信息。
    
	\section{香农模型}
	
	
	
	
	
	\section*{参考文献}
    1.C. SHANNON - Communication Theory of Secrecy Systems B.S.T.J. V o l . 28, October,
    1949, p. 656.\par
    2.M. MACHTEY, P. YOUNG - An introductlon to the general Theory of algorrthms -
    North-Holland, 1978.\par
    3.G. BRASSARD - A note on the complexlty of cryptography - I E E E Trans. on 1.". ,
    Vol. IT-25,n0 2 , hrch 1979, p. 232.\par
    4.W. DIFFIE, M. HELLEMAN - New directions in cryptography, I E E E TrdnS. on I.T..
    V o l . IT-22, no 6, Sovember 1976, p. 644.\par
    5.A . LEMPEL - Cryptoloqy in tr3nsition. Computinq Surveys, V o l . 11, n o 4, mcember
    1979, p. 285 (Example. p. 300).\par
    6.A . KOLMCGOROV - Thrre dppronchrs t o tht-. qudntitativr drfinition of information.
    Problemy Pecedaci Informacii 1 , 4-7, 1965.\par
    7.G. CHAITIN - Alqorrthmlc Information Throry - IBM J. RPS. Drv. Vol. 21, July 1977,p.350.\par
    8.T. FINE - Theories of Probability. Academic Press, 1973 (Chapter V) .\par
    9.A. LEMPEL, J. ZIV - On the complexity o€ sequences - I E E E Trans. on I.T.,
    Vol. IT-22, no 1. January 1976, p. 75.\par
    10.E. FISHER - Measuring Cryptoqrahic performance with production Processes.
    Cryptologia, V o l . 5, no 3, July 1981, p. 158\par
\end{document}

